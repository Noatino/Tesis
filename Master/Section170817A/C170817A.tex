\chapter{Una ventana hac�a la astrof�sica multimensajera: \newline WG170817/GRB170817A}

%El 17 de Agosto del 2017 el monitor de destellos de rayos gamma (GBM)\footnote{Por sus siglas en  ingl�s;  Gamma-Ray Burst Monitor} a bordo del sat�lite espacial FERMI  dispar� la alerta a las 12:41:06.47 UT \citep{2017GCN.21520....1V} de la detecci�n del GRB170817A la cual fue coincidente espacial y temporalmente (con un retraso de $\sim$2 s) con la detecci�n de ondas gravitacionales por parte de los interfer�metros LIGO y Virgo \citep{GCNGW...LW,GCNGW...LW2}. El hecho de que estas dos alarmas coincidieran significar�a la primera detecci�n de una kilonova, la contraparte electromagn�tica de ondas gravitacionales. A partir de este momento se inicio una campa�a intensiva de telescopios en distintas longitudes de onda \citep{2017ApJ...848L..12A}. La primer detecci�n en �ptico por telescopios terrestres fue realizada por el observatorio  LBT\footnote{Por sus siglas en  ingl�s; Large Binocular Telescope Observatory} \citep{2017GCN.21520....2V}. Nueve d�as despu�s se detecta emisi�n de rayos X \textbf{[??]} en la posici�n de la onda gravitacional 

El 17 de Agosto de 2017 a las 12:41:06.47 UT el monitor de rayos gamma (GBM)\footnote{Por sus siglas en  ingl�s;  Gamma-Ray Burst Monitor} a bordo del sat�lite espacial FERMI disparar�a la alerta del Destello de Rayos Gamma  (de ahora en adelante GRB) corto GRB170817A  \citep{2017GCN.21520....1V}. Apuntaba a ser otro de los $\sim$2 GRBs detectados cada semana, generando la GCN correspondiente 14 segundos despu�s del destello. Seis minutos despu�s, en la tierra, el interferometro LIGO \footnote{(Por sus siglas en ingl�s, Laser Interferometer Gravitational-wave Observatory)} en Hanford, aparec�a un candidato a onda gravitacional en \textit{latencia baja}. Este candidato tomo relevancia debido a que era consistente con un evento coalescente de dos estrellas de neutrones y coincidente con una diferencia de $\sim$ 2 segundos del GRB170817A \citep{GCNGW...LW,GCNGW...LW2}.

 En la figura \ref{Fig1} se muestra la fracci�n del cielo en los cuales Fermi-GBM (combinando con datos de INTEGRAL para poder disminuir la incertidumbre) y LIGO (Considerando datos de Hanford y Livingston as� como de VIRGO) detectaron la onda gravitacional. Finalmente la oportuna detecci�n en la banda del �ptico por el LBT\footnote{Por sus siglas en  ingl�s; Large Binocular Telescope Observatory} \citep{2017GCN.21520....2V} permite tener certeza de la posici�n del GRB as� como de la galaxia anfitriona; NGC 4993 \citep{2017Sci...358.1556C}, la cual se encuentra a una distancia de 40 Mpc de la Tierra. Sin lugar a dudas este evento ha sido hasta el momento uno de los m�s cercanos registrados hasta la fecha y destinado a marcar un parte aguas dentro de la astrof�sica.
 
 \begin{figure}
  \centering
    \includegraphics[width=0.9\textwidth]{./Figures/apjlaa91c9f1_hr.jpg}
      \caption{A picture of the GW. Imagen tomada de \citep{2017ApJ...848L..12A} \label{Fig1}}
\end{figure}

\section{Campa�as de observaci�n}
Las observaciones 24 hrs. post merger fue de gran importancia, ya sea, tratando de poder determinar una posible galaxia anfitriona as� como de poder descartar si el afterglow detectado podr�a estar correlacionado a otro fen�meno. As� como de la eventual detecci�n de rayos-X como la emisi�n en radio. En la figura \ref{Fig2} se muestra una linea del tiempo de las observaciones a partir de la detecci�n de la onda gravitacional.

\subsection{Detecci�n de ondas gravitacionales}

\begin{figure}
  \centering
    \includegraphics[width=0.9\textwidth]{./Figures/apjlaa91c9f2_hr.jpg}
      \caption{A picture of the GW. Imagen tomada de \citep{2017ApJ...848L..12A} \label{Fig2}}
\end{figure}


