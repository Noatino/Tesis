\chapter{Introducci�n}

El descubrimiento de los Destellos de Rayos Gamma (GRBs de ahora en adelante) se produce de manera fortuita por medio del sat�lite militar VELA usado por los Estados Unidos para monitorear actividad nuclear dentro y fuera de la atm�sfera terrestre \citep{vedrenne2009gamma}. La detecci�n  de 16 destellos cortos de fotones con energ�as entre 0.2-1.5 MeV entre Julio de 1969 y Julio de 1972 por Vela \citep{1973ApJ...182L..85K} cuyo origen extraterrestre de est� radiaci�n es confirmado hasta 1973 por primera vez se hace publica la detecci�n de los destellos de rayos gamma. En la figura \ref{fig:Vela-Detection} se muestra uno de los 16 destellos detectados por Vela. En ella se aprecia dos periodos de tiempo. De manera lineal el background precedente al destello, inmediatamente se muestra una escala logar�tmica que ilustra el breve destello, los cuales pueden durar desde unos cuantos segundos hasta centenas de segundos. Adem�s de su periodo corto de existencia, son completamente impredecibles, no existe una direcci�n preferencial de la detecci�n de estos fen�menos y llegan a ser tan brillantes como cualquier otra fuente de rayos gamma.

%A poco menos del 50 aniversario de est� detecci�n estos objetos mantienen una gran cantidad de respuestas sin resolver. No es si no hasta los a�os 90 del siglo pasado

\begin{figure}[t!]
  \centerline{
  \includegraphics[width=1.0\linewidth]{Figures/Vela-Detection.png}
}
  \caption{GRB detectado el 22 de Agosto de 1970 por tres detectores a bordo del sat�lite Vela. Las flechas indican la estructura en com�n en los tres detectores. Imagen tomada de \cite{1973ApJ...182L..85K}.}
  \label{fig:Vela-Detection}
\end{figure}

Estos fenomenos poseen distintas...